\documentclass[10pt,a4paper]{article}
\usepackage[margin=.5in]{geometry}
\usepackage[utf8]{inputenc}
\usepackage[T1]{fontenc}
\usepackage{dot2texi}
\usepackage{tikz}
\usetikzlibrary{shapes,arrows}\usepackage{amsmath}
\usepackage{amsfonts}
\usepackage{amssymb}
\usepackage{graphicx}
\usepackage{verbatim}
\usepackage{listings}
\usepackage{color}
\usepackage{parskip}
\usepackage{float}
\usepackage{mathtools}
% \usepackage{minted}
\usepackage{verbatim}
\usepackage{listings}
\usepackage{optidef}
\usepackage{bm}
\usepackage{algorithm}
\usepackage{algpseudocode}
\usepackage{hyperref}
\usepackage{cleveref}
\usepackage{enumerate}
\restylefloat{figure}
% \setminted{fontsize=\tiny,linenos=true,frame=single,breaklines=true}
\allowdisplaybreaks

\algrenewcommand{\Return}{\State\algorithmicreturn~}
\DeclareMathOperator*{\argmin}{arg\,min}

\title{IPM forumalations for thesis}
\author{Pratyai Mazumder}

\renewcommand{\thesubsection}{\thesection.\roman{subsection}}
\newcommand{\R}{\mathbb{R}}
\newcommand{\Rgz}{\R_{>0}}
\newcommand{\Rgez}{\R_{\geq0}}
\newcommand{\Rlz}{\R_{<0}}
\newcommand{\Rlez}{\R_{\leq0}}
\newcommand{\Oo}{\mathcal{O}}

\begin{document}

\maketitle

\section{Original Problem}

\subsection{Primal}

\begin{mini}|s|
{\mathbf{x}}{\mathbf{c}^T\mathbf{x}}{}{}
\addConstraint{\mathbf{A}\mathbf{x}}{= \mathbf{b}}
\addConstraint{\mathbf{x}}{\leq \bm{u}}
\addConstraint{\mathbf{x}}{\in \Rgez^m}
\end{mini}

\subsection{Primal Standard Form}

\begin{mini}|s|
{\mathbf{x}}{\mathbf{c}^T\mathbf{x}}{}{}
\addConstraint{\mathbf{A}\mathbf{x}}{= \mathbf{b}}
\addConstraint{\mathbf{x} + \mathbf{x}_u}{= \bm{u}}
\addConstraint{\mathbf{x}, \mathbf{x}_u}{\in \Rgez^m, \Rgez^m}
\end{mini}

\subsection{Dual}

\begin{maxi}|s|
{\mathbf{y}, \mathbf{y}_u}{\mathbf{b}^T\mathbf{y} - \mathbf{u}^T\mathbf{y}_u}{}{}
\addConstraint{\mathbf{A}^T\mathbf{y} - \mathbf{y}_u}{\leq \mathbf{c}}
\addConstraint{\mathbf{y}, \mathbf{y}_u}{\in \R^n, \Rgez^m}
\end{maxi}

\subsection{Dual Standard form}

\begin{maxi}|s|
{\mathbf{y}, \mathbf{y}_u}{\mathbf{b}^T\mathbf{y} - \mathbf{u}^T\mathbf{y}_u}{}{}
\addConstraint{\mathbf{A}^T\mathbf{y} + \mathbf{y}_u + \mathbf{z}}{= \mathbf{c}}
\addConstraint{\mathbf{y}, \mathbf{y}_u, \mathbf{z}}{\in \R^n, \Rgez^m, \Rgez^m}
\end{maxi}

But if we want $\mathbf{y}$ variables to be free, then dualize the standard from, then standardize again:
\begin{maxi}|s|
{\mathbf{y}, \mathbf{y}_u}{\mathbf{b}^T\mathbf{y} + \mathbf{u}^T\mathbf{y}_u}{}{}
\addConstraint{\mathbf{A}^T\mathbf{y} + \mathbf{y}_u + \mathbf{z}_1}{= \mathbf{c}}
\addConstraint{\mathbf{y}_u + \mathbf{z}_2}{= \bm{0}}
\addConstraint{\mathbf{y}, \mathbf{y}_u, \mathbf{z}_1, \mathbf{z}_2}{\in \R^n, \R^m, \Rgez^m, \Rgez^m}
\end{maxi}

\section{Solver Forms}

\subsection{Long Step Path Following Method}

Ref: \href{https://link.springer.com/book/10.1007/978-0-387-40065-5}{Numerical Optimization (Alg. 14.2)}

Due to the formulation, we have to work with the standard form.

\subsubsection{Primal}

\begin{mini}|s|
{}{c^Tx}{}{}
\addConstraint{Ax}{= b}
\addConstraint{x}{\in \Rgez^n}
\end{mini}

$$A = \begin{bmatrix}\mathbf{A} & \bm{0} \\ \mathbf{I} & \mathbf{I}\end{bmatrix} ~~~|~~~ x = \begin{bmatrix}\mathbf{x} \\ \mathbf{x}_u\end{bmatrix} ~~~|~~~ b = \begin{bmatrix}\mathbf{b} \\ \mathbf{u}\end{bmatrix} ~~~|~~~ c = \begin{bmatrix}\mathbf{c} \\ \bm{0}\end{bmatrix}$$

\subsubsection{Dual}

\begin{maxi}|s|
{}{b^Ty}{}{}
\addConstraint{A^T\lambda + s}{= c}
\addConstraint{\lambda, s}{\in \R^m, \Rgez^n}
\end{maxi}

$$A = \begin{bmatrix}\mathbf{A} & \bm{0} \\ \mathbf{I} & \mathbf{I}\end{bmatrix} ~~~|~~~ \lambda = \begin{bmatrix}\mathbf{y} \\ \mathbf{y}_u\end{bmatrix} ~~~|~~~ b = \begin{bmatrix}\mathbf{b} \\ \mathbf{u}\end{bmatrix} ~~~|~~~ c = \begin{bmatrix}\mathbf{c} \\ \bm{0}\end{bmatrix} ~~~|~~~ s = \begin{bmatrix}\mathbf{z}_1 \\ \mathbf{z}_2\end{bmatrix} = \begin{bmatrix}\mathbf{s} \\ \mathbf{s}_u\end{bmatrix}$$

\subsubsection{Big KKT}

$$\begin{bmatrix}
0 & A^T & I \\
A & 0 & 0 \\
S & 0 & X
\end{bmatrix}
\begin{bmatrix}\delta_x \\ \delta_y \\ \delta_s\end{bmatrix} = 
\begin{bmatrix}-r_d \\ -r_p \\ -r_c\end{bmatrix}$$

\subsubsection{Small KKT}

$$\begin{bmatrix}-X^{-1}S & A^T \\ A & 0\end{bmatrix}
\begin{bmatrix}\delta_x \\ \delta_y\end{bmatrix} = 
\begin{bmatrix}-r_d + X^{-1}r_c \\ -r_p\end{bmatrix}$$
$$\delta_s = -X^{-1}(r_c + Sd_x)$$

\subsubsection{No KKT}

Let $M := AS^{-1}XA^T$ (note: this is \emph{not necessarily} a Laplacian). Then,
\begin{align*}
\delta_y &= M^+(-r_p - AS^{-1}Xr_d + AS^{-1}r_c) \\
\delta_x &= -S^{-1}(r_c - X(r_d + A^Td-y))\\
\delta_s &= -X^{-1}(r_c + Sd_x)
\end{align*}
Observe that solving the $Md_y = -r_p - AS^{-1}Xr_d + AS^{-1}r_c$ system is difficult --- even a fast Laplacian solver cannot directly help. The other inversions of diagonal matrices and multiplications with sparse or diagonal matrices can be done relatively easily (i.e. $\Oo(nnz)$).

\subsubsection{Approximate KKT}

Substituting in the $Md_y = -r_p - AS^{-1}Xr_d + AS^{-1}r_c = -r_p -  AS^{-1}(Xr_d - r_c)$ system:
\begin{align*}
\begin{bmatrix}\mathbf{A} & 0 \\ I & I\end{bmatrix}
\begin{bmatrix}\mathbf{S}^{-1} & 0 \\ 0 & \mathbf{S}_u^{-1}\end{bmatrix}
\begin{bmatrix}\mathbf{X} & 0 \\ 0 & \mathbf{X}_u\end{bmatrix}
\begin{bmatrix}\mathbf{A}^T & I \\ 0 & I\end{bmatrix}
\begin{bmatrix}d_y \\ d_{yu}\end{bmatrix} &=
-\begin{bmatrix}r_p \\ r_{pu}\end{bmatrix}
- \begin{bmatrix}\mathbf{A} & 0 \\ I & I\end{bmatrix}
\begin{bmatrix}\mathbf{S}^{-1} & 0 \\ 0 & \mathbf{S}_u^{-1}\end{bmatrix}
\Big(
	\begin{bmatrix}\mathbf{X} & 0 \\ 0 & \mathbf{X}_u\end{bmatrix}
	\begin{bmatrix}r_d \\ r_{du}\end{bmatrix}
	+ \begin{bmatrix}r_c \\ r_{cu}\end{bmatrix}
\Big) \\
\intertext{Simplifying:}
\begin{bmatrix}
	\mathbf{A}\mathbf{S}^{-1}\mathbf{X}\mathbf{A}^T & \mathbf{A}\mathbf{S}^{-1}\mathbf{X} \\
	\mathbf{S}^{-1}\mathbf{X}\mathbf{A}^T & \mathbf{S}^{-1}\mathbf{X}+\mathbf{S}_u^{-1}\mathbf{X}_u
\end{bmatrix}
\begin{bmatrix}d_y \\ d_{yu}\end{bmatrix} &=
\begin{bmatrix}
	-r_p - \mathbf{A}\mathbf{S}^{-1}(\mathbf{X}r_d - r_c) \\
	-r_{pu} - \mathbf{S}^{-1}(\mathbf{X}r_d - r_c) - \mathbf{S}_u^{-1}(\mathbf{X}_ur_{du} - r_{cu})
\end{bmatrix} \\
\intertext{Let $\mathbf{K} := \mathbf{S}^{-1}\mathbf{X}+\mathbf{S}_u^{-1}\mathbf{X}_u$ and $r_q := -r_{pu} - \mathbf{S}^{-1}(\mathbf{X}r_d - r_c) - \mathbf{S}_u^{-1}(\mathbf{X}_ur_{du} - r_{cu})$:}
\begin{bmatrix}
	\mathbf{A}\mathbf{S}^{-1}\mathbf{X}\mathbf{A}^T & \mathbf{A}\mathbf{S}^{-1}\mathbf{X} \\
	\mathbf{S}^{-1}\mathbf{X}\mathbf{A}^T & \mathbf{K}
\end{bmatrix}
\begin{bmatrix}d_y \\ d_{yu}\end{bmatrix} &=
\begin{bmatrix}
	-r_p - \mathbf{A}\mathbf{S}^{-1}(\mathbf{X}r_d - r_c) \\
	r_q
\end{bmatrix} \\
\intertext{Pretending that $\mathbf{K}$ is invertible, and scaling:}
\begin{bmatrix}
	\mathbf{A}\mathbf{S}^{-1}\mathbf{X}\mathbf{A}^T & \mathbf{A}\mathbf{S}^{-1}\mathbf{X} \\
	\mathbf{K}^{-1}\mathbf{S}^{-1}\mathbf{X}\mathbf{A}^T & I
\end{bmatrix}
\begin{bmatrix}d_y \\ d_{yu}\end{bmatrix} &=
\begin{bmatrix}
	-r_p - \mathbf{A}\mathbf{S}^{-1}(\mathbf{X}r_d - r_c) \\
	\mathbf{K}^{-1}r_q
\end{bmatrix} \\
\intertext{Eliminating $d_{yu}$:}
\begin{bmatrix}
	\mathbf{A}(\mathbf{S}^{-1}\mathbf{X} - \mathbf{S}^{-1}\mathbf{X}\mathbf{K}^{-1}\mathbf{S}^{-1}\mathbf{X})\mathbf{A}^T & 0 \\
	\mathbf{K}^{-1}\mathbf{S}^{-1}\mathbf{X}\mathbf{A}^T & I
\end{bmatrix}
\begin{bmatrix}d_y \\ d_{yu}\end{bmatrix} &=
\begin{bmatrix}
	-r_p - \mathbf{A}\mathbf{S}^{-1}(\mathbf{X}r_d - r_c + \mathbf{X}\mathbf{K}^{-1}r_q) \\
	\mathbf{K}^{-1}r_q
\end{bmatrix} \\
\intertext{Let $\mathbf{D} := \mathbf{S}^{-1}\mathbf{X} - \mathbf{S}^{-1}\mathbf{X}\mathbf{K}^{-1}\mathbf{S}^{-1}\mathbf{X} = \mathbf{S}^{-1}\mathbf{X}(I - \mathbf{K}^{-1}\mathbf{S}^{-1}\mathbf{X})$ and $\mathbf{L} := \mathbf{A}\mathbf{D}\mathbf{A}^T$ and $r_s := -r_p - \mathbf{A}\mathbf{S}^{-1}(\mathbf{X}r_d - r_c + \mathbf{X}\mathbf{K}^{-1}r_q)$:}
\begin{bmatrix}
	\mathbf{L} & 0 \\
	\mathbf{K}^{-1}\mathbf{S}^{-1}\mathbf{X}\mathbf{A}^T & I
\end{bmatrix}
\begin{bmatrix}d_y \\ d_{yu}\end{bmatrix} &=
\begin{bmatrix}
	r_s \\
	\mathbf{K}^{-1}r_q
\end{bmatrix} \\
\intertext{Pretending that $\mathbf{L}$ is invertible:}
\begin{bmatrix}
	I & 0 \\
	\mathbf{K}^{-1}\mathbf{S}^{-1}\mathbf{X}\mathbf{A}^T & I
\end{bmatrix}
\begin{bmatrix}d_y \\ d_{yu}\end{bmatrix} &=
\begin{bmatrix}
	\mathbf{L}^{-1}r_s \\
	\mathbf{K}^{-1}r_q
\end{bmatrix} \\
\intertext{Eliminating $d_y$,}
\delta_y = \begin{bmatrix}
	I & 0 \\
	0 & I
\end{bmatrix}
\begin{bmatrix}d_y \\ d_{yu}\end{bmatrix} &=
\begin{bmatrix}
	\mathbf{L}^{-1}r_s \\
	\mathbf{K}^{-1}(r_q - \mathbf{S}^{-1}\mathbf{X}\mathbf{A}^Td_y)
\end{bmatrix} \\
\intertext{And we can apply the remaining formula from \texttt{No KKT}:}
\delta_x &= -S^{-1}(r_c - X(r_d + A^Td-y))\\
\delta_s &= -X^{-1}(r_c + Sd_x)
\end{align*}

\end{document}