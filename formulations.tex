\documentclass[10pt,a4paper]{article}
\usepackage[margin=.7in]{geometry}
\usepackage[utf8]{inputenc}
\usepackage[T1]{fontenc}
\usepackage{dot2texi}
\usepackage{tikz}
\usetikzlibrary{shapes,arrows}\usepackage{amsmath}
\usepackage{amsfonts}
\usepackage{amssymb}
\usepackage{graphicx}
\usepackage{verbatim}
\usepackage{listings}
\usepackage{color}
\usepackage{parskip}
\usepackage{float}
\usepackage{mathtools}
% \usepackage{minted}
\usepackage{verbatim}
\usepackage{listings}
\usepackage{optidef}
\usepackage{bm}
\usepackage{algorithm}
\usepackage{algpseudocode}
\usepackage{hyperref}
\usepackage{cleveref}
\usepackage{enumerate}
\restylefloat{figure}
% \setminted{fontsize=\tiny,linenos=true,frame=single,breaklines=true}
\allowdisplaybreaks

\algrenewcommand{\Return}{\State\algorithmicreturn~}
\DeclareMathOperator*{\argmin}{arg\,min}

\title{IPM forumalations for thesis}
\author{Pratyai Mazumder}

\renewcommand{\thesubsection}{\thesection.\roman{subsection}}
\newcommand{\R}{\mathbb{R}}
\newcommand{\Rgz}{\R_{>0}}
\newcommand{\Rgez}{\R_{\geq0}}
\newcommand{\Rlz}{\R_{<0}}
\newcommand{\Rlez}{\R_{\leq0}}

\begin{document}

\maketitle

\section{Original Problem}

\subsection{Primal}

\begin{mini}|s|
{\mathbf{x}}{\mathbf{c}^T\mathbf{x}}{}{}
\addConstraint{\mathbf{A}\mathbf{x}}{= \mathbf{b}}
\addConstraint{\mathbf{x}}{\leq \bm{u}}
\addConstraint{\mathbf{x}}{\in \Rgez^m}
\end{mini}

\subsection{Primal Standard Form}

\begin{mini}|s|
{\mathbf{x}}{\mathbf{c}^T\mathbf{x}}{}{}
\addConstraint{\mathbf{A}\mathbf{x}}{= \mathbf{b}}
\addConstraint{\mathbf{x} + \mathbf{x}_u}{= \bm{u}}
\addConstraint{\mathbf{x}, \mathbf{x}_u}{\in \Rgez^m, \Rgez^m}
\end{mini}

\subsection{Dual}

\begin{maxi}|s|
{\mathbf{y}, \mathbf{y}_u}{\mathbf{b}^T\mathbf{y} - \mathbf{u}^T\mathbf{y}_u}{}{}
\addConstraint{\mathbf{A}^T\mathbf{y} - \mathbf{y}_u}{\leq \mathbf{c}}
\addConstraint{\mathbf{y}, \mathbf{y}_u}{\in \R^n, \Rgez^m}
\end{maxi}

\subsection{Dual Standard form}

\begin{maxi}|s|
{\mathbf{y}, \mathbf{y}_u}{\mathbf{b}^T\mathbf{y} - \mathbf{u}^T\mathbf{y}_u}{}{}
\addConstraint{\mathbf{A}^T\mathbf{y} + \mathbf{y}_u + \mathbf{z}}{= \mathbf{c}}
\addConstraint{\mathbf{y}, \mathbf{y}_u, \mathbf{z}}{\in \R^n, \Rgez^m, \Rgez^m}
\end{maxi}

But if we want $\mathbf{y}$ variables to be free, then dualize the standard from, then standardize again:
\begin{maxi}|s|
{\mathbf{y}, \mathbf{y}_u}{\mathbf{b}^T\mathbf{y} + \mathbf{u}^T\mathbf{y}_u}{}{}
\addConstraint{\mathbf{A}^T\mathbf{y} + \mathbf{y}_u + \mathbf{z}_1}{= \mathbf{c}}
\addConstraint{\mathbf{y}_u + \mathbf{z}_2}{= \bm{0}}
\addConstraint{\mathbf{y}, \mathbf{y}_u, \mathbf{z}_1, \mathbf{z}_2}{\in \R^n, \R^m, \Rgez^m, \Rgez^m}
\end{maxi}

\section{Solver Forms}

\subsection{Long Step Path Following Method}

Ref: \href{https://link.springer.com/book/10.1007/978-0-387-40065-5}{Numerical Optimization (Alg. 14.2)}

Due to the formulation, we have to work with the standard form.

\subsubsection{Primal}

\begin{mini}|s|
{}{c^Tx}{}{}
\addConstraint{Ax}{= b}
\addConstraint{x}{\in \Rgez^n}
\end{mini}

$$A = \begin{bmatrix}\mathbf{A} & \bm{0} \\ \mathbf{I} & \mathbf{I}\end{bmatrix} ~~~|~~~ x = \begin{bmatrix}\mathbf{x} \\ \mathbf{x}_u\end{bmatrix} ~~~|~~~ b = \begin{bmatrix}\mathbf{b} \\ \mathbf{u}\end{bmatrix} ~~~|~~~ c = \begin{bmatrix}\mathbf{c} \\ \bm{0}\end{bmatrix}$$

\subsubsection{Dual}

\begin{maxi}|s|
{}{b^Ty}{}{}
\addConstraint{A^T\lambda + s}{= c}
\addConstraint{\lambda, s}{\in \R^m, \Rgez^n}
\end{maxi}

$$A = \begin{bmatrix}\mathbf{A} & \bm{0} \\ \mathbf{I} & \mathbf{I}\end{bmatrix} ~~~|~~~ \lambda = \begin{bmatrix}\mathbf{y} \\ \mathbf{y}_u\end{bmatrix} ~~~|~~~ b = \begin{bmatrix}\mathbf{b} \\ \mathbf{u}\end{bmatrix} ~~~|~~~ c = \begin{bmatrix}\mathbf{c} \\ \bm{0}\end{bmatrix} ~~~|~~~ s = \begin{bmatrix}\mathbf{z}_1 \\ \mathbf{z}_2\end{bmatrix}$$

\subsection{Big KKT}

$$\begin{bmatrix}
0 & A^T & I \\
A & 0 & 0 \\
S & 0 & X
\end{bmatrix}
\begin{bmatrix}d_x \\ d_y \\ d_s\end{bmatrix} = 
\begin{bmatrix}-r_d \\ -r_p \\ -r_c\end{bmatrix}$$

\subsection{Small KKT}

$$\begin{bmatrix}-X^{-1}S & A^T \\ A & 0\end{bmatrix}
\begin{bmatrix}d_x \\ d_y\end{bmatrix} = 
\begin{bmatrix}-r_d + X^{-1}r_c \\ -r_p\end{bmatrix}$$
$$d_s = -X^{-1}(r_c + Sd_x)$$

\subsection{No KKT}

Let $L := AS^{-1}XA^T$. Then,
\begin{align*}
d_y &= L^+(-r_p - AS^{-1}Xr_d + AS^{-1}r_c) \\
d_x &= -S^{-1}(r_c - X(r_d + A^Td-y))\\
d_s &= -X^{-1}(r_c + Sd_x)
\end{align*}

\end{document}